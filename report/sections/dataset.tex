\chapter{Dataset Description}\label{ch:dataset-description}

The Forest Cover Type dataset consists of tree observations collected from four major regions of the Roosevelt National Forest in Colorado.
Each observation represents a 30 × 30 meter plot of land, and with over half a million samples, the dataset provides a rich and detailed view of forest landscapes.
All the features in this dataset are cartographic variables, meaning they come from ground-based measurements rather than satellite or remote-sensing data.

The dataset captures a variety of environmental characteristics such as elevation, slope, aspect, soil type, distance to hydrology or roadways, hillshade values, and other topographical factors.
It also includes indicators for wilderness areas and soil classifications, both represented using binary one-hot encoded columns.
These variables help describe how terrain and environmental conditions influence the type of forest vegetation present.

The task is to predict the forest cover type for each 30 × 30 meter plot.
The ground-truth labels were provided by the US Forest Service (USFS) Region 2 Resource Information System (RIS), which contains expert-verified vegetation data.
Most of the independent variables originate from the US Geological Survey (USGS) and USFS resources.
All features are provided in their raw numerical form, and no scaling or normalization is applied beforehand.

This combination of physical terrain measurements and vegetation labels makes the dataset ideal for exploring how environmental factors contribute to forest composition, and it provides a strong foundation for testing a variety of machine learning models.

The features are grouped as follows:

\begin{enumerate}
    \item \textbf{Topographical / Terrain Features}
    \begin{itemize}
        \item \textbf{Elevation}
        Height of the plot above sea level (in meters).
        Different trees grow at different altitudes due to temperature and oxygen differences.
        \item \textbf{Aspect}
        Direction the slope faces (0–360°).
        Sunlight exposure affects soil moisture and vegetation patterns.
        \item \textbf{Slope}
        Steepness of the terrain (in degrees).
        Steeper areas may limit tree growth or influence water runoff.
    \end{itemize}
    \item \textbf{Distance-Based Features}
    \begin{itemize}
        \item \textbf{Horizontal\_Distance\_To\_Hydrology}
        Horizontal distance to the nearest water source (streams, lakes).
        Trees differ in their water requirements.
        \item \textbf{Vertical\_Distance\_To\_Hydrology}
        Elevation difference relative to nearby water bodies.
        Lower/sunken areas may hold more moisture.
        \item \textbf{Horizontal\_Distance\_To\_Roadways}
        Distance to the nearest road.
        Roads can alter soil, introduce human disturbance, or influence forest composition.
        \item \textbf{Horizontal\_Distance\_To\_Fire\_Points}
        Distance to known wildfire ignition points.
        Fire-tolerant species may dominate closer regions.
    \end{itemize}
    \item \textbf{Hillshade Features}
    \begin{itemize}
        \item \textbf{Hillshade\_9am}
        Amount of sunlight at 9 AM, given terrain orientation.
        \item \textbf{Hillshade\_Noon}
        Sunlight at solar noon, usually the brightest time.
        \item \textbf{Hillshade\_3pm}
        Sunlight at 3 PM.
    \end{itemize}
    \item \textbf{Wilderness Area Indicators}
    The dataset includes four wilderness areas, encoded as: Wilderness\_Area1, Wilderness\_Area2, Wilderness\_Area3, Wilderness\_Area4.
    Each is a binary (0/1) feature that indicates whether a plot belongs to that specific protected area.
    Different wilderness regions have different regulations, climates, soil conditions, and vegetation patterns-which directly influence forest type.
    \item \textbf{Soil Type Indicators}
    There are 40 soil types, represented as: Soil\_Type1, Soil\_Type2, …, Soil\_Type40.
    Each is a binary feature indicating the presence of a specific soil classification.
    Soil characteristics such as composition, drainage, pH, and nutrient availability strongly affect which tree species can grow.
\end{enumerate}

%\importTableFigure{tables/data_dataset_statistics.csv}{Dataset Summary Statistics}{data_dataset_statistics}
%
%\importTableFigure{tables/data_dataset_info.csv}{Dataset Information (Structure Overview)}{data_dataset_info}
%
%\importTableFigure{tables/data_dataset_dtypes.csv}{Feature Data Types}{data_dataset_dtypes}
%
%\importTableFigure{tables/data_dataset_describe.csv}{Statistical Summary of Features}{data_dataset_describe}

\chapter{Introduction}\label{ch:introduction}



Forests play a crucial role in maintaining ecological balance, supporting biodiversity, and influencing global climate patterns.
Understanding forest composition is important for conservation, land management, wildfire prevention, and ecological research.
One of the key aspects of forest analysis is determining the type of forest cover present in a given region.
Forest cover types are influenced by a mix of environmental factors such as soil properties, elevation, climate, and topographic conditions.
These features interact in complex ways, and even small variations can result in shifts between different types of vegetation.
Because of this complexity, accurately identifying forest cover types is a challenging but essential task for researchers, ecologists, and environmental agencies.

Traditional methods of determining forest cover-such as manual field surveys or rule-based ecological models-tend to rely on predefined thresholds or human expertise.
While useful, these methods often fail to capture nonlinear relationships and subtle interactions in environmental data.
In contrast, modern machine learning (ML) techniques provide a powerful way to analyze large-scale ecological datasets and detect patterns that may not be obvious using classical approaches.
ML algorithms can learn from multidimensional environmental features and produce accurate predictions about forest categories, making them highly valuable for applications in land-use planning, habitat conservation, and environmental monitoring.

Machine learning has already shown strong potential in environmental science, particularly in areas like crop classification, remote sensing analysis, wildlife detection, and climate modeling.
ML models can process large datasets collected from sensors, satellites, and field measurements to reveal hidden structures in the data.
For forest ecosystems, where multiple factors simultaneously influence vegetation growth, ML provides a systematic way to classify cover types and detect environmental trends.
With forests facing increasing pressures from climate change, wildfires, and human activity, data-driven systems for monitoring cover types can help decision-makers plan better interventions and manage resources efficiently.

The motivation for this project comes from the need to build a reliable prediction system that can classify forest cover types based on environmental attributes.
The dataset used in this study includes quantitative measurements of soil characteristics, elevation, slope, aspect, distance to hydrology and roadways, and wilderness areas, all of which influence forest composition.
Since the dataset is purely numerical and relatively large, it offers a good opportunity to explore different preprocessing steps and modeling strategies.
The first phase involved conducting an Exploratory Data Analysis (EDA) to understand data distributions, correlations, and feature importance.
Visualizations such as histograms, box plots, and basic scatter-plots helped identify patterns and detect potential issues such as skewed values.

Preprocessing steps such as scaling numerical features and checking for outliers were performed to prepare the data for model training.
Because this is a multiclass classification problem involving seven forest cover categories, choosing the right evaluation metrics and handling class imbalance were important considerations.
Standardization was applied to numerical features to ensure that models such as logistic regression, SVM, and neural networks could perform optimally.
Since there are no categorical features in this dataset, encoding methods were not required.

The goal of this project was to build and compare multiple machine learning models using the algorithms taught in class.
These included Logistic Regression, Support Vector Machines, Neural Networks, K-Means, Agglomerative Clustering, DBSCAN, and Gaussian Mixture Models.
Each model was trained individually, and their performances were evaluated using appropriate accuracy metrics for multiclass classification.
Hyperparameters were tuned to ensure that every model was tested fairly.
Cluster-based models were also included for unsupervised analysis, allowing us to explore how well the data naturally groups into cover types without labels.

The dataset was split into training and testing sets to evaluate how well each model generalizes to unseen data.
After training all models, their results were compared, and the strengths and weaknesses of each approach were analyzed.
Some models performed better because they handled high-dimensional, continuous data more effectively, while others struggled due to the complexity of the classification task.
The comparison helped illustrate how different algorithms behave on real-world environmental datasets.

Beyond model accuracy, the broader importance of this project lies in its practical applications.
Accurate forest cover classification can support environmental conservation, wildfire planning, sustainable forestry, and ecosystem research.
By developing data-driven models, we can help automate the monitoring process and contribute to a better understanding of ecological patterns.
Machine learning can thus serve as a useful tool for environmental scientists, enabling large-scale analysis that would be difficult to achieve manually.


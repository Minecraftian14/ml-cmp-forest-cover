%! Author = Anirudh Sharma, Shreya Gupta
%! Date = 09-12-2025
%! Mail Addresses = mcxiv14@gmail.com, shreya.gupta.0624@gmail.com

\documentclass[12pt,a4paper]{report}

% =======================
% MODULAR SETTINGS
% =======================
\usepackage[utf8]{inputenc}
\usepackage[T1]{fontenc}
\usepackage{lmodern}
\usepackage[english]{babel}
\usepackage{graphicx}
\usepackage{amsmath, amssymb, amsfonts}
\usepackage{booktabs}
%\usepackage{hyperref}
\usepackage[colorlinks=true, urlcolor=blue, linkcolor=black]{hyperref}
\usepackage{fancyhdr}
\usepackage{geometry}
\usepackage{xcolor}
\usepackage{titlesec}
\usepackage{setspace}

\usepackage{csvsimple}

\usepackage{microtype}

% =======================
% FONTS
% =======================
%\usepackage{fontspec}
%\setmainfont{TeX Gyre Termes}

\onehalfspacing

\input{settings/layout}
\input{settings/header}
\input{settings/macros}

% =======================
% BIBLIOGRAPHY
% =======================
\usepackage[backend=biber,style=ieee]{biblatex} % IEEE style
\addbibresource{references.bib}

% =======================
% DOCUMENT
% =======================
\begin{document}

% -----------------------
% Title Page
% -----------------------
{
    \let\clearpage\relax
    \input{sections/titlepage}
}

% -----------------------
% Abstract
% -----------------------
{
    \let\clearpage\relax
    \chapter*{Abstract}
    \addcontentsline{toc}{chapter}{Abstract}
% \input{sections/abstract}

    \begin{singlespace}
        In this project, I worked with two different classification datasets: one for predicting whether a person is a
        smoker based on biosignals, and another for identifying forest cover types from environmental features.
        The main goal was to apply all the machine learning models that we learned in class, including Logistic Regression,
        SVM, Neural Networks, K-Means, Agglomerative Clustering, DBSCAN, and Gaussian Mixture Models.
    \end{singlespace}

    \begin{singlespace}
        For both datasets, I first performed data preprocessing and exploratory data analysis to understand the structure of the data and the relationships between different features.
        Then I trained each model separately and evaluated their performance using appropriate metrics for binary and multiclass classification.
        Finally, I compared all the models to see which ones worked the best and tried to explain the reasons behind their performance.
    \end{singlespace}

    \begin{singlespace}
        Overall, the project helped me understand how different algorithms behave on different kinds of datasets, how important preprocessing is, and how model performance can vary depending on the complexity of the data.
    \end{singlespace}

    \href{https://github.com/Minecraftian14/ml-cmp-forest-cover}{\underline{\textcolor{blue}{https://github.com/Minecraftian14/ml-cmp-forest-cover}}}

}

    % -----------------------
    % Table of Contents
    % -----------------------
    \tableofcontents
    \newpage

    % -----------------------
    % Chapters
    % -----------------------
    \chapter{Introduction}\label{ch:introduction}



Forests play a crucial role in maintaining ecological balance, supporting biodiversity, and influencing global climate patterns.
Understanding forest composition is important for conservation, land management, wildfire prevention, and ecological research.
One of the key aspects of forest analysis is determining the type of forest cover present in a given region.
Forest cover types are influenced by a mix of environmental factors such as soil properties, elevation, climate, and topographic conditions.
These features interact in complex ways, and even small variations can result in shifts between different types of vegetation.
Because of this complexity, accurately identifying forest cover types is a challenging but essential task for researchers, ecologists, and environmental agencies.

Traditional methods of determining forest cover-such as manual field surveys or rule-based ecological models-tend to rely on predefined thresholds or human expertise.
While useful, these methods often fail to capture nonlinear relationships and subtle interactions in environmental data.
In contrast, modern machine learning (ML) techniques provide a powerful way to analyze large-scale ecological datasets and detect patterns that may not be obvious using classical approaches.
ML algorithms can learn from multidimensional environmental features and produce accurate predictions about forest categories, making them highly valuable for applications in land-use planning, habitat conservation, and environmental monitoring.

Machine learning has already shown strong potential in environmental science, particularly in areas like crop classification, remote sensing analysis, wildlife detection, and climate modeling.
ML models can process large datasets collected from sensors, satellites, and field measurements to reveal hidden structures in the data.
For forest ecosystems, where multiple factors simultaneously influence vegetation growth, ML provides a systematic way to classify cover types and detect environmental trends.
With forests facing increasing pressures from climate change, wildfires, and human activity, data-driven systems for monitoring cover types can help decision-makers plan better interventions and manage resources efficiently.

The motivation for this project comes from the need to build a reliable prediction system that can classify forest cover types based on environmental attributes.
The dataset used in this study includes quantitative measurements of soil characteristics, elevation, slope, aspect, distance to hydrology and roadways, and wilderness areas, all of which influence forest composition.
Since the dataset is purely numerical and relatively large, it offers a good opportunity to explore different preprocessing steps and modeling strategies.
The first phase involved conducting an Exploratory Data Analysis (EDA) to understand data distributions, correlations, and feature importance.
Visualizations such as histograms, box plots, and basic scatter-plots helped identify patterns and detect potential issues such as skewed values.

Preprocessing steps such as scaling numerical features and checking for outliers were performed to prepare the data for model training.
Because this is a multiclass classification problem involving seven forest cover categories, choosing the right evaluation metrics and handling class imbalance were important considerations.
Standardization was applied to numerical features to ensure that models such as logistic regression, SVM, and neural networks could perform optimally.
Since there are no categorical features in this dataset, encoding methods were not required.

The goal of this project was to build and compare multiple machine learning models using the algorithms taught in class.
These included Logistic Regression, Support Vector Machines, Neural Networks, K-Means, Agglomerative Clustering, DBSCAN, and Gaussian Mixture Models.
Each model was trained individually, and their performances were evaluated using appropriate accuracy metrics for multiclass classification.
Hyperparameters were tuned to ensure that every model was tested fairly.
Cluster-based models were also included for unsupervised analysis, allowing us to explore how well the data naturally groups into cover types without labels.

The dataset was split into training and testing sets to evaluate how well each model generalizes to unseen data.
After training all models, their results were compared, and the strengths and weaknesses of each approach were analyzed.
Some models performed better because they handled high-dimensional, continuous data more effectively, while others struggled due to the complexity of the classification task.
The comparison helped illustrate how different algorithms behave on real-world environmental datasets.

Beyond model accuracy, the broader importance of this project lies in its practical applications.
Accurate forest cover classification can support environmental conservation, wildfire planning, sustainable forestry, and ecosystem research.
By developing data-driven models, we can help automate the monitoring process and contribute to a better understanding of ecological patterns.
Machine learning can thus serve as a useful tool for environmental scientists, enabling large-scale analysis that would be difficult to achieve manually.


    \chapter{Methodology}\label{ch:methodology}



A typical machine learning workflow usually moves step-by-step: we load the data, explore it, clean and preprocess it, train different models, and then produce the final predictions or outputs.

But in this project, instead of following this straight, one-directional path, I used a more flexible and cycle-based approach inspired by how real data science pipelines are built.

One of the main changes is how visualizations are used.
Rather than treating plots and graphs as something done only at the beginning during EDA, I included visualization throughout the entire process.
This means I kept checking the data during preprocessing, during feature transformations, and even while evaluating models.
Doing this made it easier to catch unusual patterns early and helped me understand how each model responds to different features.

Another important part of the methodology is modularity.
Since many of the models require similar preprocessing steps or repeated experiment setups, it made sense to organize the code using helper functions and reusable components.
This not only reduced repetition but also made the experiments more structured and easier to compare.
Having these small utility methods allowed me to run tests more smoothly and keep the workflow clean and consistent across different models.


%\importPlotFigure{figures/plot_Model Comparison.png}{Model Comparison}{model_time_comparison}


\section{Dataset Description}\label{sec:dataset-description}

The Forest Cover Type dataset used in this project is provided in a clean and ready-to-use CSV format, which makes the loading process very simple.
Since the file already contains structured numerical features with no complicated formatting issues, we can import it directly into a DataFrame without requiring any special parsing.
For this project, the dataset is loaded into a variable named \texttt{ds\_source}.

Unlike some Kaggle competitions, this project does not involve generating submission files, so there is no separate \texttt{ds\_test} dataset.
All analysis, preprocessing, and model training are performed on the single dataset provided.

After loading the data, one of the first steps is to rename several columns.
Many of the original feature names are long, especially the distance-related and hillshade features, which makes them harder to read in tables and nearly impossible to display neatly in plots.
To improve visualization quality and keep the graph labels clean, the following renaming scheme is applied:

\begin{itemize}
    \item Elevation => Elev
    \item Aspect => Aspc
    \item Slope => Slpe
    \item Horizontal\_Distance\_To\_Hydrology => HDTH
    \item Vertical\_Distance\_To\_Hydrology => VDTH
    \item Horizontal\_Distance\_To\_Roadways => HDTR
    \item Hillshade\_9am => H09A
    \item Hillshade\_Noon => H12P
    \item Hillshade\_3pm => H03P
    \item Horizontal\_Distance\_To\_Fire\_Points => HDFP
    \item Wilderness\_AreaN => WAN
    \item Soil\_TypeN => STNN
\end{itemize}

These shorter names make plots cleaner, tables more compact, and the overall analysis easier to follow.
While the change is small, it improves consistency across visualizations and helps maintain readability throughout the project.


\section{Exploratory Data Analysis (EDA)}\label{sec:exploratory-data-analysis}

The goal of the Exploratory Data Analysis stage is to carefully examine the dataset for any missing values, unusual patterns, extreme observations, or other statistical irregularities.
Even though this stage may look routine, it serves a much deeper purpose than simply producing charts-it helps reveal important characteristics of the data that directly affect how preprocessing and modeling decisions are made.

To build a solid understanding of the dataset, multiple visualizations and descriptive statistics are used.
These include distribution plots, correlation checks, and feature comparisons.
Each plot is chosen deliberately so that it adds something meaningful to our interpretation, rather than being included just for decoration.
By analyzing these visual patterns and summary statistics, we gain clearer insights into how the features behave, which ultimately guides how we clean the data and prepare it for the machine learning models.


\section{Data Preprocessing \& Feature Engineering}\label{sec:data-preprocessing}
The data preprocessing and feature engineering stage acts as the foundation of the entire machine learning pipeline.
This is where the raw dataset is converted into a clean, structured, and meaningful form that models can understand and learn from.
Every transformation applied here is based directly on insights collected during the EDA phase, ensuring that the preprocessing choices are data-driven and not arbitrary.

\subsection{Transformation Strategy}\label{subsec:transformation-strategy}
The main goal of this stage is to improve both the interpretability and predictive value of the features while keeping the dataset compatible with different types of models.
Since linear models, neural networks, and tree-based algorithms each have different expectations about input data, the preprocessing strategy is designed to work well across all of them.

\textbf{Numerical Transformations}
Most numerical features in the dataset (except \textit{Aspc}) showed a skewed Gaussian-like distribution.
To correct this and make the features more symmetric, we apply a PowerTransformer.
PowerTransformer uses the Yeo-Johnson transformation, which is similar to Box-Cox but works even when the data contains zeros or negative values.
The goal is to stabilize variance and make the distribution closer to normal, which helps many models learn more effectively.

\textbf{Categorical Transformations}
The dataset already comes with its categorical attributes (Wilderness Areas and Soil Types) fully one-hot encoded, so no additional encoding steps are needed.
This is convenient because it ensures immediate compatibility with all models.

\subsection{Scaling and Standardization}\label{subsec:scaling-and-standardization}
To prevent features with larger magnitudes from dominating others, normalization is essential.
After applying the PowerTransformer, the numerical features are already scaled to behave like standardized variables.

Meanwhile, the categorical one-hot encoded columns naturally lie in the 0-1 range, so no further scaling is necessary.
This creates a balanced feature space suitable for both linear and non-linear models.


\section{Setups and Helpers}\label{sec:setups-and-helpers}
Although this section does not directly appear in the final report output, it plays a foundational role within the notebook implementation.
Executed immediately after the import statements, it handles several preliminary configurations essential for smooth experimentation.
These include:

\begin{itemize}
    \item Initialization of the notification and logging system
    \item PyPlot and visualization styling adjustments
    \item Random seed management for reproducibility
    \item Notebook display and formatting controls
    \item Model training, saving, and submission automation
    \item Definition and registration of POP operations
\end{itemize}


\section{Model Training and Evaluation}\label{sec:model-training-and-evaluation}
The final step of this stage involves constructing the models, tuning their hyperparameters, and evaluating their performance.
A consistent training framework is used across all algorithms, but with room for model-specific adjustments when needed.
This balance allows fair comparison while still allowing each model to perform at its best.

\subsection{Data Splitting}\label{subsec:data-splitting}
Before model training, the fully preprocessed dataset is divided into training (96%) and validation (4%) sets. Stratified sampling is used to maintain the original class proportions in both subsets, ensuring fair evaluation and preventing class imbalance issues during validation.

\subsection{Model Selection and Training}\label{subsec:model-selection-and-training}
A diverse suite of models was implemented to evaluate various learning paradigms and understand how different algorithms behave on the Forest Cover Type dataset.
Each model provides a unique perspective on the data and contributes to the overall comparative study:

\begin{itemize}
    \item Logistic Regression - Served as the baseline linear classifier.
    It was tested with different regularization strategies (L1, L2) and solver options to assess how well linear decision boundaries can separate the classes.
    \item Support Vector Machine (SVM) - Explored both linear and non-linear kernels to capture more complex decision boundaries.
    Despite being computationally heavier, SVM helps evaluate margin-based classification behavior.
    \item Neural Networks - Implemented as a multi-layer perceptron to model non-linear relationships.
    It provides insight into how well deep feature interactions can improve predictive power.
    \item K-Means Clustering - Used in an unsupervised setting to examine natural grouping patterns within the dataset.
    This helps analyze whether forest cover types form well-separated clusters in feature space.
    \item Agglomerative Clustering - Offers a hierarchical perspective, allowing us to observe how clusters merge progressively and whether any meaningful structure emerges.
    \item DBSCAN - Helps detect density-based patterns and potential anomalies.
    It is particularly useful for understanding irregular or noise-prone regions in the data.
    \item Gaussian Mixture Models (GMM) - Evaluates whether probabilistic soft clustering aligns with the actual class distribution and highlights overlapping feature regions.
\end{itemize}

\subsection{Hyperparameter Optimization}\label{subsec:hyperparameter-optimization}
To ensure that each model performs at its best, Grid Search was used to systematically explore multiple combinations of hyperparameters.
Grid Search exhaustively evaluates every possible parameter setting defined in the search space.

This is combined with Cross-Validation, where the training set is split into multiple folds.
Each model is trained on a subset of the folds and validated on the remaining one, cycling through all folds.

This process ensures that the chosen hyperparameters generalize well and are not overfitted to a single train-validation split.
It also improves the reliability and fairness of comparisons across models.

\subsection{Evaluation Metrics}\label{subsec:evaluation-metrics}
Since this is a multiclass classification problem, a single score is not enough to judge performance.
Therefore, several metrics were used to provide a complete picture:
\begin{itemize}
    \item Precision – Measures how many of the predicted samples for a class are actually correct.
    \item Recall – Measures how many actual samples of a class were successfully identified.
    \item F1-Score – The harmonic mean of precision and recall, useful when dealing with class imbalance or uneven performance across classes.
    \item Confusion Matrix – Provides a detailed breakdown of correct and incorrect predictions for every class, helping visualize where models struggle.
\end{itemize}
These metrics allow fine-grained evaluation and make it easier to compare models not just by accuracy but by their behavior on individual forest cover types.




    \chapter{Dataset Description}\label{ch:dataset-description}

The Forest Cover Type dataset consists of tree observations collected from four major regions of the Roosevelt National Forest in Colorado.
Each observation represents a 30 × 30 meter plot of land, and with over half a million samples, the dataset provides a rich and detailed view of forest landscapes.
All the features in this dataset are cartographic variables, meaning they come from ground-based measurements rather than satellite or remote-sensing data.

The dataset captures a variety of environmental characteristics such as elevation, slope, aspect, soil type, distance to hydrology or roadways, hillshade values, and other topographical factors.
It also includes indicators for wilderness areas and soil classifications, both represented using binary one-hot encoded columns.
These variables help describe how terrain and environmental conditions influence the type of forest vegetation present.

The task is to predict the forest cover type for each 30 × 30 meter plot.
The ground-truth labels were provided by the US Forest Service (USFS) Region 2 Resource Information System (RIS), which contains expert-verified vegetation data.
Most of the independent variables originate from the US Geological Survey (USGS) and USFS resources.
All features are provided in their raw numerical form, and no scaling or normalization is applied beforehand.

This combination of physical terrain measurements and vegetation labels makes the dataset ideal for exploring how environmental factors contribute to forest composition, and it provides a strong foundation for testing a variety of machine learning models.

The features are grouped as follows:

\begin{enumerate}
    \item \textbf{Topographical / Terrain Features}
    \begin{itemize}
        \item \textbf{Elevation}
        Height of the plot above sea level (in meters).
        Different trees grow at different altitudes due to temperature and oxygen differences.
        \item \textbf{Aspect}
        Direction the slope faces (0–360°).
        Sunlight exposure affects soil moisture and vegetation patterns.
        \item \textbf{Slope}
        Steepness of the terrain (in degrees).
        Steeper areas may limit tree growth or influence water runoff.
    \end{itemize}
    \item \textbf{Distance-Based Features}
    \begin{itemize}
        \item \textbf{Horizontal\_Distance\_To\_Hydrology}
        Horizontal distance to the nearest water source (streams, lakes).
        Trees differ in their water requirements.
        \item \textbf{Vertical\_Distance\_To\_Hydrology}
        Elevation difference relative to nearby water bodies.
        Lower/sunken areas may hold more moisture.
        \item \textbf{Horizontal\_Distance\_To\_Roadways}
        Distance to the nearest road.
        Roads can alter soil, introduce human disturbance, or influence forest composition.
        \item \textbf{Horizontal\_Distance\_To\_Fire\_Points}
        Distance to known wildfire ignition points.
        Fire-tolerant species may dominate closer regions.
    \end{itemize}
    \item \textbf{Hillshade Features}
    \begin{itemize}
        \item \textbf{Hillshade\_9am}
        Amount of sunlight at 9 AM, given terrain orientation.
        \item \textbf{Hillshade\_Noon}
        Sunlight at solar noon, usually the brightest time.
        \item \textbf{Hillshade\_3pm}
        Sunlight at 3 PM.
    \end{itemize}
    \item \textbf{Wilderness Area Indicators}
    The dataset includes four wilderness areas, encoded as: Wilderness\_Area1, Wilderness\_Area2, Wilderness\_Area3, Wilderness\_Area4.
    Each is a binary (0/1) feature that indicates whether a plot belongs to that specific protected area.
    Different wilderness regions have different regulations, climates, soil conditions, and vegetation patterns-which directly influence forest type.
    \item \textbf{Soil Type Indicators}
    There are 40 soil types, represented as: Soil\_Type1, Soil\_Type2, …, Soil\_Type40.
    Each is a binary feature indicating the presence of a specific soil classification.
    Soil characteristics such as composition, drainage, pH, and nutrient availability strongly affect which tree species can grow.
\end{enumerate}

%\importTableFigure{tables/data_dataset_statistics.csv}{Dataset Summary Statistics}{data_dataset_statistics}
%
%\importTableFigure{tables/data_dataset_info.csv}{Dataset Information (Structure Overview)}{data_dataset_info}
%
%\importTableFigure{tables/data_dataset_dtypes.csv}{Feature Data Types}{data_dataset_dtypes}
%
%\importTableFigure{tables/data_dataset_describe.csv}{Statistical Summary of Features}{data_dataset_describe}

    \chapter{Exploratory Data Analysis}\label{ch:exploratory-data-analysis}

This section presents a detailed exploratory data analysis of the dataset.
We examine data completeness, outlier presence, feature distributions, and sensitivity to extreme values.
Each subsection summarizes visual analyses and their corresponding insights.

%----------------------------------------------------------------


\section{Missing Value Detection}\label{sec:missing-value-detection}
\importPlotFigure{figures/plot_Missing Value Counts.png}{Missing Value Counts}{eda_missing_counts}
\textbf{Observation:} There is not a single missing value in any of the columns.\\
\textbf{Inference:} The dataset is well-structured and complete.
No imputation, removal, or flagging of missing values is required.

%----------------------------------------------------------------


\section{Outlier Detection}\label{sec:outlier-detection}
\importPlotFigure{figures/plot_Outlier Counts.png}{Outlier Counts}{eda_outlier_counts}
\textbf{Observation:} \textit{VDTH} and a few others exhibit a small number of values beyond the interquartile range (IQR).\\
\textbf{Inference:} Since the dataset is beyond 500 k samples, a small number of outliers such as 173 wont matter.

%----------------------------------------------------------------


\section{Duplication Detection}\label{sec:duplication-detection}
\textbf{Observation:} There is not even a single duplicate row in the dataset.\\
\textbf{Inference:} The dataset is clean in this regard, and there is no need to drop any of the rows.

%----------------------------------------------------------------


\section{Feature Distribution (Numerical)}\label{sec:feature-distribution-numerical}
\importPlotFigure{figures/plot_Feature Distribution (Numerical).png}{Numerical Features Distribution}{feature_distribution_numerical}
\textbf{Observation:}
\begin{itemize}
    \item \textbf{Simplest Gaussians:} \textit{Elev, Slpe, VDTH, H03P}
    \item \textbf{Skewed Gaussians:}   \textit{HDTH, H09A, H12P, HDFP}
    \item \textbf{Not Gaussians:}      \textit{Aspc, HDTR}
\end{itemize}
\textbf{Inference:} In either case, they can do good with some level of power transformations.

%----------------------------------------------------------------


\section{Feature Distribution (Wilderness)}\label{sec:feature-distribution-wilderness}
\importPlotFigure{figures/plot_Feature Distribution (Wilderness).png}{Wilderness Features Distribution}{feature_distribution_wilderness}
\textbf{Observation:}
\begin{itemize}
    \item \textit{Wilderness Area 1 and 3} is well-balanced.
    \item \textit{Wilderness Area 2 and 4} are highly skewed.
\end{itemize}
\textbf{Inference:}
Wilderness Area 2 and 4 still have a std of 0.2, which is still nearly half of what Wilderness Area 1.
No need to process this feature further.

%----------------------------------------------------------------


\section{Feature Distribution (Soil Type)}\label{sec:feature-distribution-soil-type}
\importPlotFigure{figures/plot_Feature Distribution (Soil Type).png}{Soil Type Features Distribution}{feature_distribution_soil_type}
\textbf{Observation:} Every soil type is pretty well distributed.
\textbf{Inference:} No actions needed.

%----------------------------------------------------------------


\section{Feature Spread (Numerical)}\label{sec:feature-spread-numerical}
\importPlotFigure{figures/plot_Feature Spread (Numerical).png}{Numerical Feature Spread}{feature_spread_numerical}

\textbf{Observation:}
\begin{enumerate}
    \item \textbf{Differing Scales}
    \begin{itemize}
        \item \textit{Slpe} has a small scale (1e0:1)
        \item \textit{Hillshade features} have medium scales (1e0:2)
        \item \textit{HDTR, and HDFP} have large scales (1e0:3)
    \end{itemize}
    \item \textbf{Skewness}
    \begin{itemize}
        \item \textit{Slpe, HDTH, HDTR, and HDFP} all show a concentration of data at the lower end.
        \item \textit{H09A and H12P}are bunched towards the top.
    \end{itemize}
    \item \textbf{Outliers}
    \begin{itemize}
        \item \textit{VDTH} shows a massive cluster of outliers on the upper end.
        \item \textit{H12P} There are distinct outliers at the very bottom.
    \end{itemize}
\end{enumerate}

\textbf{Inference:}
\begin{itemize}
    \item It would be a good idea to standardize the features to introduce stability and prevent overshadowing of the features with smaller scales.
    \item Will apply power transform to adjust the skewed-ness of the various features identified above.
    \item As verified earlier, the outlier count is not very high; therefore, they can be set aside.
\end{itemize}

%----------------------------------------------------------------


\section{Outlier Sensitivity (Numerical)}\label{sec:outlier-sensitivity-(numerical)}
\importPlotFigure{figures/plot_Outlier Sensitivity (Numerical).png}{Numerical Outlier Sensitivity}{outlier_sensitivity_numerical}

\textbf{Observation:}
\begin{itemize}
    \item \textbf{Extreme Invariance:} \textit{HDTR and HDFP} show exponentially differentiated presence compared to other features.
    \item \textbf{Negligible Invariance:} \textit{Slpe and H03P} have no offset to show at all, likely to be the most sensitive to outliers.
    \item In general, all features lie beyond the threshold.
\end{itemize}
\textbf{Inference:} A standard scaler could likely solve all issues.


%----------------------------------------------------------------


\section{Correlation Matrix}\label{sec:correlation-matrix}
\importPlotFigure{figures/plot_Correlation Matrix Main Dataset.png}{Correlation Matrix}{correlation}

\textbf{Observation:}
\begin{enumerate}
    \item Most features show a moderate correlation, between -.5 to +.5.
    \item \textit{H03P, and H09A} show some level of negative correlation, around -0.75.
    \item Other notable relations are shown by \textit{Elev and Slpe}, \textit{Aspc and HDTR}.
\end{enumerate}

\textbf{Inference:}
The relationships between the Hillshade features indicate overlapping information.
However, they don't stand on absolute indifference either.
Since none of the features are strongly related, let the models create any associations if necessary.

%----------------------------------------------------------------


\section{Principal Components}\label{sec:principal-components}
\importPlotFigure{figures/plot_3D PCA Projection.png}{Principal Components}{pca}

\textbf{Observation:}
The PCA projection shows a large degree of overlap among most classes, with no clearly separated clusters.
All the forest cover categories blend into one another, indicating that their feature distributions are similar in the reduced 3D space.

\textbf{Inference:} This overlap suggests that the variables contributing to cover type classification are continuous and interrelated, making sharp boundaries between classes difficult to capture with PCA. It also implies that while PCA captures overall variance, it may not fully separate cover type categories-nonlinear methods (e.g., t-SNE or UMAP) might reveal clearer class distinctions.

%----------------------------------------------------------------


\section*{Summary of EDA Findings}
The dataset exhibits strong completeness and reasonable diversity across both physical and behavioral variables.
While outliers are present in \textit{Age} and \textit{NCP}, most features show stable distributions.
The observations suggest that feature scaling and selective outlier handling will enhance model robustness without major data loss.


%    \input{sections/preprocessing}
%    \chapter{Model Training and Evaluation}\label{ch:model-training-and-evaluations}

Following the data preprocessing and feature engineering stages, the next step was to train a set of machine learning models to predict the forest cover type for each 30×30 meter land segment.
The goal was to explore a wide range of classifiers-spanning simple linear methods to more complex non-linear and probabilistic approaches-to understand how different learning paradigms perform on this multi-class, terrain-based dataset.
By evaluating multiple algorithm families, we gain a clearer picture of which modeling strategies are best suited for the environmental patterns present in the forest cover data.


\section{Experimental Setup and Data Splitting}\label{sec:experimental-setup-and-data-splitting}
To ensure robust evaluation, the dataset was divided into training and testing subsets using a stratified split to maintain class balance across all partitions.

\begin{itemize}
    \item 96\% (557771 samples) of the data was allocated for model development (training and validation), and
    \item 4\% (23241) was held out as a final test set for unbiased evaluation.
\end{itemize}

During model training, Grid Search was paired with K-Fold Cross-Validation (using k values between 2 and 5) to get more dependable estimates of model performance, especially during hyperparameter tuning.
This strategy reduces the risk of overfitting and gives a better sense of how well the model generalizes across different portions of the data.

However, because the dataset is extremely large (around 550 thousand samples), running an exhaustive grid search becomes computationally impractical on a local machine.
As a result, the search space had to be carefully narrowed down-focusing only on the most impactful hyperparameters and limiting the number of combinations explored-to make the tuning process possible.


\section{Data Pipelining}\label{sec:data_pipelining}
A lightweight data pipeline was used to keep the preprocessing steps consistent and reproducible.
Since only the numerical features required transformation, a ColumnTransformer was created to apply a PowerTransformer to the Gaussian-like features while leaving the one-hot encoded wilderness and soil type features unchanged.
Using a pipeline ensures that the same preprocessing applied during training is also consistently applied to the validation set, keeping the workflow clean and reliable.
As for the remaining features, since one hot encoding has already been applied to them, we do not have to create any more transformers.

\subsection{Log Transform}\label{subsec:log-transform}
The log transform reduces skewness by compressing large values more than small ones.
It is commonly used when data is strictly positive and has a long right tail.
However, it cannot be applied to zero or negative values.

\subsection{Box-Cox Transform}\label{subsec:box-cox-transform}
The Box-Cox transformation is a generalization of the log transform.
It can automatically choose an optimal power parameter to make the distribution more Gaussian-like.
It still requires all input values to be strictly positive.

\subsection{Yeo-Johnson Transform}\label{subsec:yeo-jognson-transform}
The Yeo-Johnson transformation is similar to Box-Cox but allows zero and negative values, making it more flexible.
It is the version used inside scikit-learn’s PowerTransformer, and it helps stabilize variance and normalize skewed numerical features.


\section{Model Selection Strategy}\label{sec:model-selection-strategy}
The project adopted a comparative modeling framework, where diverse algorithms were evaluated under a consistent preprocessing setup.
By examining multiple learning paradigms-both supervised and unsupervised-we aimed to understand how different model families interpret the forest cover data and where each method excels or struggles.

\begin{enumerate}
    \item \textbf{Supervised Classification Models}
    These algorithms directly learn to predict the forest cover type from labeled training data.
    \begin{itemize}
        \item Logistic Regression
        \item Support Vector Machine (SVM)
        \item Neural Network (MLP Classifier)
    \end{itemize}

    \item \textbf{Unsupervised Clustering Models}
    These models attempt to group the data based on structure alone, without using labels.
    They are included to explore natural separations in the feature space.
    \begin{itemize}
        \item K-Means Clustering
        \item Agglomerative (Hierarchical) Clustering
        \item DBSCAN
        \item Gaussian Mixture Models (GMM)
    \end{itemize}
\end{enumerate}

\subsection{Algorithm Overviews}\label{subsec:algorithm-overviews}

\begin{enumerate}
    \item \textbf{Logistic Regression}
    Acts as the baseline linear classifier.
    It models decision boundaries using a linear combination of features and supports regularization (L1/L2) to prevent overfitting.
    Despite its simplicity, it provides a strong benchmark for comparison.

    \importPlotFigure{figures/plot_Best LR Config.png}{Grid Search on Logistic Regression}{Best_LR_Config}
    \importPlotFigure{figures/plot_Confusion Matrix_ GSLR.png}{Confusion Matrix Logistic Regression}{ConfusionMatrix_GSLR}

    \item \textbf{Support Vector Machine (SVM)}
    SVM attempts to find the best separating hyperplane by maximizing the margin between classes.
    With different kernels (linear, RBF), it can capture both simple and moderately complex decision boundaries.

    \importPlotFigure{figures/plot_Best SV Config.png}{Grid Search on Support Vector Machine}{Best_SV_Config}
    \importPlotFigure{figures/plot_Confusion Matrix_ GSSV.png}{Confusion Matrix Support Vector Machine}{ConfusionMatrix_GSSV}

    \item \textbf{Neural Network (MLP)}
    A multi-layer perceptron that learns non-linear relationships through hidden layers and activation functions.
    Useful for capturing deeper feature interactions that linear models might miss.

    \importPlotFigure{figures/plot_Best MP Config.png}{Grid Search on Neural Network}{Best_MP_Config}
    \importPlotFigure{figures/plot_Confusion Matrix_ GSMP.png}{Confusion Matrix Neural Network}{ConfusionMatrix_GSMP}

    \item \textbf{K-Means Clustering}
    Partitions of the dataset into k clusters by minimizing within-cluster variance.
    Helpful for exploring whether forest types form distinct geometric groups.

    \importPlotFigure{figures/plot_Best KM Config.png}{Grid Search on K-Means Clustering}{Best_KM_Config}
    \importPlotFigure{figures/plot_Confusion Matrix_ GSKM.png}{Confusion Matrix K-Means Clustering}{ConfusionMatrix_GSKM}

    \item \textbf{Agglomerative Clustering}
    A hierarchical, bottom-up clustering method that repeatedly merges the closest clusters.
    It provides structural insight into how data points group together at different similarity thresholds.

    \importPlotFigure{figures/plot_Best AC Config.png}{Grid Search on Agglomerative Clustering}{Best_AC_Config}
    \importPlotFigure{figures/plot_Confusion Matrix_ GSAC.png}{Confusion Matrix Agglomerative Clustering}{ConfusionMatrix_GSAC}

    \item \textbf{DBSCAN}
    A density-based algorithm that groups together points in high-density regions while marking isolated points as noise.
    Useful for identifying irregular or non-spherical clusters.

    \importPlotFigure{figures/plot_Best DS Config.png}{Grid Search on DBSCAN}{Best_DS_Config}
    \importPlotFigure{figures/plot_Confusion Matrix_ GSDS.png}{Confusion Matrix DBSCAN}{ConfusionMatrix_GSDS}

    \item \textbf{Gaussian Mixture Models (GMM)}
    A probabilistic clustering method assuming data is generated from overlapping Gaussian distributions.
    Unlike K-means, it supports soft clustering and can capture complex cluster shapes.

    \importPlotFigure{figures/plot_Best GM Config.png}{Grid Search on Gaussian Mixture Models}{Best_GM_Config}
    \importPlotFigure{figures/plot_Confusion Matrix_ GSGM.png}{Confusion Matrix Gaussian Mixture Models}{ConfusionMatrix_GSGM}

\end{enumerate}


\section{Hyperparameter Optimization}\label{sec:hyperparameter-optimization2}
Hyperparameter tuning played a critical role in enhancing model performance.

\begin{itemize}
    \item Regularization strength (for logistic regression),
    \item \textit{Logistic Regression:} Regularization Type (penalty) and Factor (C), Maximum number of iterations.
    \item \textit{Support Vector Machine:} Kernel, Degree, Maximum number of iterations.
    \item \textit{Neural Networks:} Hidden Layer Count and Sizes, Maximum number of iterations.
    \item \textit{K-Means:} Number of Clusters, Maximum number of iterations.
    \item \textit{Agglomerative Clustering:} No hyperparameters (Number of final clusters formed is kept the same as the number of targets).
    \item \textit{Gaussian Mixture Modeling:} Maximum number of iterations.
    \item \textit{DBScan:} Epsilon, Min Samples.
\end{itemize}


\section{Model Evaluation Metrics}\label{sec:model-evaluation-metrics}

Model performance was primarily evaluated using Accuracy, but since the dataset is not perfectly balanced, relying on accuracy alone would not give a complete picture.
To address this, additional metrics such as Precision, Recall, and F1-Score were monitored during cross-validation to better understand class-wise performance and identify which forest cover types were harder for the models to distinguish.
The evaluation setup provided:
\begin{itemize}
    \item Cross-validation scores (mean ± standard deviation),
    \item Final performance on the stratified test split,
    \item Detailed metric summaries for comparing different model configurations.
\end{itemize}

To interpret prediction patterns more clearly, confusion matrices were generated, allowing us to inspect misclassifications between specific cover types.
These visualizations were particularly helpful for understanding overlap among similar vegetation categories.


\section{Comparative Analysis and Observations}\label{sec:comparative-analysis-and-observations}
The performance comparison between supervised and unsupervised models revealed clear distinctions in how well each category handled the forest cover classification task.

\begin{itemize}
    \item \textbf{Supervised Models (LR, SVM, NN):}\\
    Since supervised models learn directly from labeled examples, they consistently achieved higher accuracy and more stable predictions across all cross-validation folds.
    Overall, supervised models had a significant advantage because they explicitly use the ground-truth cover type labels during training, allowing them to learn class-specific patterns that unsupervised algorithms cannot access.

    \item \textbf{Unsupervised Models (KM, AC, DS, GM):}\\
    Unsupervised models, by design, operate without label information, and therefore their performance was noticeably weaker for the actual classification task.
    Instead of predicting forest cover types, they attempt to group samples based solely on feature similarities.
    These results are expected: without access to the true forest cover labels, unsupervised models have no way to learn class boundaries, making them unsuitable as primary predictors.
\end{itemize}

%    \chapter{Results and Discussion}\label{ch:results-and-discussion}

This chapter summarizes how the different models performed on the forest cover dataset, how preprocessing influenced the results, and what insights were gained from observing the behavior of both supervised and unsupervised algorithms.


\section{Quantitative Results}\label{sec:quantitative-results}
All supervised models were evaluated using stratified 2 to 5 fold cross-validation, followed by testing on the held-out test split.
Metrics such as accuracy, precision, recall, and F1-score were averaged across folds for consistency.
In general:

\begin{itemize}
    \item Neural Networks achieved the highest accuracy among supervised models.
    \item Logistic Regression lagged behind because the feature–target relationships were largely non-linear.
    \item SVM performed was significantly slower due to dataset size.
    \item Unsupervised models showed noticeably poorer alignment with the true cover type labels, which is expected since they do not use label information.
\end{itemize}

A simple ranking based on accuracy placed Neural Networks first, followed by Logistic Regression, then Support Vector Machine.
Clustering models did not produce competitive class-label predictions but were still useful for exploring natural group structures in the data.

Confusion matrices revealed that certain forest types were harder to separate, particularly those with similar elevations or soil compositions.

\importPlotFigure{figures/plot_Model Accuracy Comparison.png}{Model Accuracy Comparison}{model_accuracy_comparison}


\section{Impact of Preprocessing and Feature Engineering}\label{sec:impact-of-preprocessing-and-feature-engineering}

\subsection{Feature Transformations}\label{subsec:feature-transformations}
PowerTransformer significantly improved the symmetry of numerical features.
This benefited linear models and SVM the most, where accuracy increased by a noticeable margin.
Neural networks were less sensitive but still showed more stable training behavior after transformation.

\subsection{Outlier Handling}\label{subsec:outlier-handling}
Outliers were retained because they seemed to reflect natural variation in terrain measurements.
Removing them did not improve performance and occasionally worsened generalization.

\subsection{Class Imbalance}\label{subsec:class-imbalance}
Although some classes had fewer samples, stratified splitting helped preserve their proportional presence during training.
Since no oversampling was applied, metrics like recall and F1-score helped highlight classes that were harder to detect.

\subsection{Feature Encoding}\label{subsec:feature-encoding}
The dataset already arrived with one-hot encoded wilderness and soil type features.
This worked well for all supervised models and avoided unnecessary preprocessing overhead.


\section{Model Behavior and Interpretations}\label{sec:model-behavior-and-interpretations}

\subsection{Linear Models (Logistic Regression)}\label{subsec:linear-models-(logistic-regression)}
Logistic Regression was fast and easy to interpret but struggled due to the strong non-linearity in the dataset.
Terrain and soil features interact in complex ways that simple linear boundaries cannot capture.

\subsection{Kernel-Based Models (SVM)}\label{subsec:kernel-based-models-(svm)}
SVM could not handle the dataset any better, might require higher degrees.
Unfortunately, due to computational limitations, that's not possible.

\subsection{Neural Network}\label{subsec:neural-network}
The neural network performed the best, likely because it can learn multiple levels of non-linear interactions across the cartographic features.
It balanced accuracy and generalization reasonably well.

\subsection{Unsupervised Models}\label{subsec:unsupervised-models}
Clustering algorithms (K-Means, Agglomerative, DBSCAN, GMM) were unable to match the performance of supervised models.
Since they receive no label information, their cluster assignments did not align well with the actual forest cover categories.
They were more useful for understanding high-level structure rather than classification.

\importPlotFigure{figures/plot_Metric Comparison over Top Models.png}{Metric Comparison over Top Models}{metric_comparison}


\section{Evaluation Metrics Beyond Accuracy}\label{sec:evaluation-metrics-beyond-accuracy}
Accuracy alone did not capture the full story due to class imbalance.
Therefore:
\begin{itemize}
    \item Precision and Recall revealed that minority classes were consistently harder to predict.
    \item F1-Score helped identify trade-offs between false positives and false negatives.
    \item Confusion matrices made misclassification patterns clear, especially for the forest types that shared similar soil or elevation values.
\end{itemize}

These additional metrics provided a more realistic view of how the models performed across all seven classes.


\section{Discussion of Findings}\label{sec:discussion-of-findings}

\subsection{Non-linearity in Terrain Data}\label{subsec:non-linearity-in-terrain-data}
The stronger performance of Neural Networks indicates that forest cover classification depends heavily on non-linear interactions among elevation, soil type, and distance-based features.

\subsection{Outlier Retention}\label{subsec:outlier-retention}
Keeping outliers likely helped the models generalize better, since these values reflect natural landscape variation rather than noise.

\subsection{Unsupervised vs. Supervised Models}\label{subsec:unsupervised-vs-supervised-models}
Clustering methods struggled not because they performed poorly, but because they were simply not designed to leverage label information.
This highlights the importance of supervised learning for structured prediction tasks like this one.

\subsection{Efficiency vs. Accuracy}\label{subsec:efficiency-vs.-accuracy}
Neural Networks offered the best accuracy but required more training time.
Logistic Regression was extremely fast but significantly less accurate.
SVM balanced both but suffered from scalability issues.


\section{Limitations}\label{sec:limitations}

\begin{itemize}
    \item Dataset imbalance affected recall in certain classes.
    \item No cost-sensitive evaluation was applied; some misclassifications may be more harmful in real ecological analysis.
    \item Neural networks remain less interpretable, making it harder to explain decisions directly.
    \item Clustering models are fundamentally misaligned with the classification goal, which limits their usefulness beyond exploratory purposes.
\end{itemize}


    % -----------------------
    % Conclusion
    % -----------------------
    {
        \chapter*{Conclusion}
        \addcontentsline{toc}{chapter}{Conclusion}
        This project successfully demonstrated the application of a structured machine learning pipeline for predicting cardiovascular disease (CVD) risk based on obesity-related parameters.
    Through systematic data exploration, feature engineering, and model evaluation, the study established a comprehensive framework that balances interpretability, performance, and robustness.

    The results highlight that ensemble-based methods, particularly XGBoost, deliver the best predictive performance, achieving an accuracy of 93.27\% on the held-out test dataset.
    This improvement was driven by careful preprocessing steps, including Box-Cox transformation for normalization, SMOTE-based oversampling for class balance, and appropriate encoding strategies for categorical variables.
    The experiments also revealed that retaining certain outliers improved model generalization, reflecting the realistic variability of health-related data.

    Comparative analyses confirmed that while linear models provided valuable interpretability, they lacked the flexibility to capture the complex non-linear interactions among features such as age, dietary habits, and activity levels.
    Tree-based and boosting algorithms, on the other hand, effectively modeled these relationships and maintained robustness against noise.

    In conclusion, this work demonstrates that thoughtfully engineered preprocessing combined with advanced ensemble learning can yield highly reliable health risk prediction models.
    The findings underscore the potential of machine learning in preventive healthcare analytics and provide a foundation for future extensions, such as integrating explainability frameworks (e.g., SHAP or LIME) and testing the model on real-world clinical datasets to enhance its applicability and trustworthiness
    }

    % -----------------------
    % References
    % -----------------------
    {
        \chapter*{References}
        \addcontentsline{toc}{chapter}{References}
        \printbibliography

        \vspace{1em}
        \section*{Web Resources}
        \addcontentsline{toc}{section}{Web Resources}

        \begin{itemize}
            \item \url{https://www.kaggle.com/competitions/ait-511-course-project-1-obesity-risk/overview}
            \item \url{https://www.kaggle.com/datasets/aravindpcoder/obesity-or-cvd-risk-classifyregressorcluster}
            \item \url{https://optuna.org/}
            \item \url{https://xgboost.readthedocs.io/en/stable/}
            \item \url{https://lightgbm.readthedocs.io/en/latest/}
            \item \url{https://scikit-learn.org/stable/}
            \item \url{https://imbalanced-learn.org/stable/references/generated/imblearn.over_sampling.SMOTE.html}
            \item \url{https://pandas.pydata.org/docs/}
            \item \url{https://numpy.org/doc/stable/}
            \item \url{https://matplotlib.org/stable/contents.html}
            \item \url{https://seaborn.pydata.org/}
            \item \url{https://docs.python.org/3/}
        \end{itemize}
    }

\end{document}
